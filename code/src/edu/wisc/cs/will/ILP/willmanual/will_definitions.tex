\usepackage{amsfonts,amsgen,amstext}
\usepackage[utf8]{inputenc}
\usepackage{graphicx,xspace,cite,tabularx,ulem}
\usepackage{subfig,caption}
\usepackage{float,algorithm,hyperref,url,listings,color}
\usepackage[noend]{algorithmic}
\usepackage{pdfpages}

\renewcommand{\algorithmicrequire}{\textbf{Input:}}
\renewcommand{\algorithmicensure}{\textbf{Output:}}
\renewcommand{\algorithmiccomment}[1]{// #1}
%\renewcommand{\algorithmicendif}{}
\newcommand{\name}{\textsc{WILL}\xspace}
\newcommand{\will}{\textsc{WILL}\xspace}
\newcommand{\onion}{\emph{Onion}\xspace}
\newcommand{\gleaner}{\textsc{Gleaner}\xspace}

\newcommand{\MLN}{\textsc{MLN}\xspace}
\newcommand{\MLNs}{\textsc{MLN}s\xspace}
\newcommand{\alc}{\textsc{Alchemy}\xspace}
\newcommand{\eat}[1]{}
\newcommand{\shout}[1]{\textbf{XXX} #1 \textbf{XXX}}
\newcommand{\filled}[1]{\textbf{YYY} #1 \textbf{YYY}}
\newcommand{\set}[1]{\ensuremath \left\{#1\right\}}
\newcommand{\rel}[1]{\mathtt{#1}}
\newcommand{\con}[1]{\text{`#1'}}
\newcommand{\imp}{=>}
\newcommand{\cost}{\mathrm{cost}}
\newcommand{\maxsat}{\textsf{MAXSAT}}
\newcommand{\avar}[1]{\mathtt{#1}} % variables in algorithms
\newcommand{\code}[1]{\texttt{#1}\xspace}
\newcommand{\menu}[1]{\textbf{\textsf{#1}}}
\newcommand{\usergui}[1]{\textsf{#1}}

% Compact itemize and enumerate.  Note that they use the same counters and
% symbols as the usual itemize and enumerate environments.
\def\compactify{\itemsep=-1pt \topsep=0pt \partopsep=-1pt \parsep=-2pt}
\let\latexusecounter=\usecounter
\newenvironment{CompactItemize}
 {\def\usecounter{\compactify\latexusecounter}
  \begin{itemize}}
 {\end{itemize}\let\usecounter=\latexusecounter}

\newcounter{examplecounter}[section]
\newenvironment{example}                        % Example environment.
  {\refstepcounter{examplecounter}\trivlist\item
    [\hskip\labelsep{\bf Example \theexamplecounter}]}%   Acts just like a theorem%    %   environment, except that
  {\endtrivlist}                                %   body is typeset in
                                                %   Roman.
\newtheorem{proposition}{Proposition}[section]

\newenvironment{mylisting}
{\begin{list}{}{\setlength{\leftmargin}{1em}}\item\bfseries\color{blue}}
{\end{list}}


\newenvironment{rfc}
{\begin{list}{}{\setlength{\leftmargin}{1em}}\item\bfseries\color{red}}
{\end{list}}

\newenvironment{mylistingNoIndent}
{\begin{list}{}{\setlength{\leftmargin}{0em}}\item\normalsize\bfseries\color{blue}}
{\end{list}}


\makeatletter
\newbox\sf@box
\newenvironment{SubFloat}[2][]%
{\def\sf@one{#1}%
\def\sf@two{#2}%
\setbox\sf@box\hbox
\bgroup}%
{ \egroup
\ifx\@empty\sf@two\@empty\relax
\def\sf@two{\@empty}
\fi
\ifx\@empty\sf@one\@empty\relax
\subfloat[\sf@two]{\box\sf@box}%
\else
\subfloat[\sf@one][\sf@two]{\box\sf@box}%
\fi}
\makeatother

\makeatletter
  \def\theHALC@line{\thealgorithm-\theALC@line}
  \def\theHALC@rem{\thealgorithm-\theALC@rem}
\makeatother

\let\origdescription\description
\renewenvironment{description}{
  \setlength{\leftmargini}{0em}
  \origdescription
  \setlength{\itemindent}{0em}
  \setlength{\labelsep}{\textwidth}
}
{\endlist} 